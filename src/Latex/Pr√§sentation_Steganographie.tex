\documentclass{beamer}
%\setmainfont[Ligatures=TeX, ]{Linux Libertine O}
%\setmainfont{Old Standard}

\usepackage{polyglossia}
\setdefaultlanguage{german}
\setotherlanguage{greek}
\setotherlanguage{english}

\usepackage{fontspec}
\setmainfont[Ligatures=TeX]{Linux Libertine O}
\newfontfamily\greekfont[Script=Greek]{Linux Libertine O}
\newfontfamily\greekfontsf[Script=Greek]{Linux Libertine O}

\usepackage{graphicx}

\usepackage{amsthm}

\AtBeginSection[]
{
  \begin{frame}
    \frametitle{Inhalt}
    \tableofcontents[currentsection]
  \end{frame}
}
\usetheme{Warsaw}
\hypersetup{pdfstartview={Fit}}

\newtheorem{mydef}{Definition}

\begin{document}

\title{Steganographie und Steganalyse auf Smartphones}
\author{Ulrich Viefhaus}
\institute
{
  FernUniversität in Hagen
}
\date{16. und 17. September 2016}
\subject{Steganographie, Inforation Hiding, Steganalysis}

\frame[allowframebreaks]{\titlepage}

\section{Grundlagen der Steganographie}

\begin{frame}
    \frametitle{Bedeutung}
    \begin{mydef}
    Das Wort Steganographie setzt sich aus den beiden altgriechischen Wörtern \textgreek[variant=ancient]{στεγανός} (bedeckt) und \textgreek[variant=ancient]{γράφειν} (schreiben) zusammen. Sinngemäß bedeutet es „geheimes Schreiben“.
  \end{mydef}
\end{frame}

\begin{frame}
    \frametitle{Abgrenzung Kryptographie}
    Kryptographie und Steganographie haben unterschiedliche Ziele.
	  \begin{itemize}
	    \item Kryptographie: Nachrichten werden durch Verschlüsselung geschützt, aber offen versendet .
	    \item Steganographie: Existenz von Nachrichten wird verborgen.
	  \end{itemize}
	  Es ist durchaus möglich eine Nachricht erst zu verschlüsseln und sie dann durch Steganographie zu verstecken.
\end{frame}
  
\begin{frame}
    \frametitle{Kerckhoffs' Prinzip}
    \begin{mydef}
      Die Sicherheit eines Verschlüsselungsverfahrens darf nur von der Geheimhaltung des Schlüssel abhängen, nicht jedoch von der Geheimhaltung des Algorithmus.
  \end{mydef}
    \begin{itemize}
      \item Wurde bei Steganographie lange nicht beachtet.
      \item Optische Unerkennbarkeit war entscheidend.
      \item Findet in neueren Algorithmen Anwendung.
    \end{itemize}
    \note{Zitat aus Woflgang Ertel -- Angewandte Kryptographie}
    \note{War lange nicht wichtig, da allein die optischen Eigenschaften Relevanz hatten.}
\end{frame}

\begin{frame}
  \frametitle{Tarnmedium}
  \begin{mydef}
  Ein Tarnmedium (engl. cover medium) ist ein Medium, in dem eine geheime Nachricht versteckt wird.
  \end{mydef}
  Beispiel: Bild im jpeg Format.
\end{frame}

\begin{frame}
\frametitle{Verdeckter Kanal}
  \begin{mydef}
  Ein verdeckter Kanal (engl. hidden channel) dient dem Übermitteln von Informationen über ein Protokoll auf eine Art, die in dem Protokoll nicht vorgesehen ist.
  \end{mydef}
  Beispiel: Nutzung des „Options“ Feldes im TCP Header zur Datenübertragung.
\end{frame}

\begin{frame}
    \frametitle{Stegoschlüssel}
    \begin{mydef}
    Ein Stegoschlüssel (engl. stegokey) ist ein Schlüssel, mit dem die Verteilung der Bits der geheimen Nachricht in dem Tarnmedium festgelegt wird.
  \end{mydef}
  Beispiel: Vereinbarter Schlüssel wird als Seed für einen PRNG genutzt, der bestimmt in welchen Pixeln eines Bildes die geheime Nachricht versteckt wird.
\end{frame}

\begin{frame}
  Analog zu kryptografischen Schlüsseln gibt es zwei Arten von Stegoschlüsseln:
  \begin{itemize}
	    \item Symmetrische Stegoschlüssel
	    \item Asymmetrische Stegoschlüssel	    
	  \end{itemize}
\end{frame}

\begin{frame}
\frametitle{Notwendige Eigenschaften}
  Bei der verdeckten Kommunikation müssen drei notwendige Eigenschaften gegeneinander abgewogen werden:
  \begin{itemize}
    \item Bandbreite (engl. steganographic bandwidth)
    \item Unerkennbarkeit (engl. security)
    \item Widerstandsfähigkeit (engl. robustness)
  \end{itemize}
  
\end{frame}

\begin{frame}  
\frametitle{Magisches Dreieck}
\centering
    \def\svgwidth{\columnwidth}
    \input{magic_triangle.pdf_tex}
\end{frame}

\section{Steganographie auf Smartphones}

\begin{frame}
  \frametitle{Besonderheiten auf Smartphones}
  \begin{itemize}
    \item Energieverbrauch ist von großer Bedeutung
    \item Vielzahl an Sensoren und Netzwerkschnittstellen
    \item Große Anzahl sensibler Daten
  \end{itemize}
\end{frame}

\begin{frame}
    \frametitle{Arten von verdeckten Kanälen auf Smartphones}
	  \begin{itemize}
	    \item Lokale Kanäle
	    \item Objektkanäle
	    \item Netzwerkkanäle
	  \end{itemize}
\end{frame}

\begin{frame}
  \frametitle{Lokale Kanäle}
  Lokale Kanäle werden durch steganographische Verfahren erzeugt, die auf Systemresourcen angewandt werden.
  Sie ermöglichen die verdeckte Kommunikation zwischen Prozessen auf dem selben Gerät.
  Beispiele:
  \begin{itemize}
    \item Vibrationseinstellungen
    \item Lockfiles
    \item Prozessorauslastung
  \end{itemize}
\end{frame}

\begin{frame}
  \frametitle{Objektkanäle}
  Objektkanäle werden durch steganographische Verfahren erzeugt, die auf virtuelle Objekte wie Dateien angewandt werden.
  Objekte eignen sich sowohl zum Speichern von geheimen Informationen als auch zum Versand über andere Dienste.
  Beispiele:
  \begin{itemize}
    \item Multimediadateien in diversen Formaten (z.B. JPEG, Gif, MPEG, mp3)
    \item Textdateien (z.B. PDF und Word)
  \end{itemize}
\end{frame}

\begin{frame}
  \frametitle{Netzwerkkanäle}
  Netzwerkkanäle werden durch steganographische Verfahren erzeugt, die auf Netzwerkprotokolle- und Anwendungen.
  Netzwerkdienste eignen sich zur verdeckten Kommunikation mit anderen Geräten.
  Beispiele:
  \begin{itemize}
    \item HTTP mittels URLs
    \item Videostream mit spezifischen Abständen zwischen Frames
    \item Einfügen von energiearmen Töne in GSM-kodierte Sprachübertragung
  \end{itemize}
\end{frame}

\section{Gefahren}
\begin{frame}
  \frametitle{Gefahren von Steganographie auf Smartphones}
  Durch Steganographie entstehen auf Smartphones zwei weitere Gefahren für den Nutzer, die nur schwer zu erkennen sind.
  \begin{itemize}
    \item Unbemerktes Ausleiten von sensiblen Informationen.
    \item Unbemerktes Empfangen von Befehlen.
  \end{itemize}
\end{frame}

\begin{frame}
  \frametitle{Schema für das Ausleiten von Informationen}
  \centering
    \def\svgwidth{8cm}
    \input{hidden_channel.pdf_tex}
\end{frame}

\begin{frame}
  \frametitle{Schema für das Empfangen von Befehlen}
  \centering
    \def\svgwidth{\columnwidth}
    \input{hidden_commands.pdf_tex}
\end{frame}

\section{Gegenmaßnahmen}

\begin{frame}
\frametitle{Gegenmaßnahmen}  
Gegenmaßnahmen sind in drei Bereiche gegliedert:
\begin{itemize}
  \item Erkennung (Stegoanalyse)
  \item Beseitigung
  \item Reduktion der Bandbreite
\end{itemize}
\end{frame}

\begin{frame}
  \frametitle{Maßnahmen gegen lokale Kanäle}
  Erkennen von Steganographie durch:
  \begin{itemize}
    \item Taint tracking
    \item Überwachung von API-Aufrufen
    \item Überprüfung auf Grenzwerte
  \end{itemize}
  Falls ein Zugriff auf lokale Ressourcen als bösartig eingeschätzt wird, kann dieser unterbunden werden.
\end{frame}

\begin{frame}
  \frametitle{Maßnahmen gegen Objektkanäle}
  Erkennen von Steganographie durch:
  \begin{itemize}
    \item Signaturen
    \item Statistischen Methoden
  \end{itemize}
  Nach der Erkennung kann die versteckte Nachricht aus dem Tarnmedium entfernt werden.
\end{frame}

\begin{frame}
  \frametitle{Maßnahmen gegen Netzwerkkanäle}
  Erkennen oder Entfernen von Steganograhie durch
  \begin{itemize}
    \item Network traffic normalizer
    \item Statistischen Methoden
    \item Maschinelles Lernen
  \end{itemize}
\end{frame}

\section{Weiterführende Literatur}
  \begin{frame}[allowframebreaks]
  \frametitle{Further reading}    
  \begin{thebibliography}{10}    
  \beamertemplatearticlebibitems
  \bibitem{BAGE}
    
  \end{thebibliography}
  \end{frame}
  
\begin{frame}
	\centering Danke für Ihre Aufmerksamkeit!
	\centering Fragen?
\end{frame}

\end{document}
